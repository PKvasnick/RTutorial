\documentclass[9pt]{beamer}
\usetheme{Darmstadt}
\usepackage[utf8]{inputenc}
\usepackage[slovak]{babel}
\usepackage{amsmath}
\usepackage{amsfonts}
\usepackage{amssymb}
\usepackage{graphicx}
\usepackage{booktabs}

\usepackage{caption}
\usepackage{subcaption}

\author{Peter Kvasnička}
\title{Úvod do R}
%\setbeamercovered{transparent} 
%\setbeamertemplate{navigation symbols}{} 
%\logo{} 
\institute{Univerzita Karlova, Praha} 
\date{Kurz pre 4. ročník BMF \newline Jeseň 2017} 
\subject{Statistics and data science} 

\beamertemplatenavigationsymbolsempty

\setbeamerfont{page number in head/foot}{size=\large}
\setbeamertemplate{footline}[frame number]

\begin{document}

\begin{frame}
\titlepage
\end{frame}

\clearpage % ================================================================
\section{Prehľad}
% ================================================================
\begin{frame}{Čo sa naučíte v tomto kurze}
	\begin{block}{1. Používať R}
		Trocha neskromné, nemáme veľa času.
		\begin{description}
			\item[Načítať dáta] import z textových súborov a Excelu
			\item[Preskúmať dáta] kreslenie - \texttt{ggplot2}
			\item[Upraviť dáta] manipulácia s dátami, súhrny, filtrovanie atď. - \texttt{tidyverse}
			\item[Analyzovať dáta] lineárny model, ANOVA atď.
			\item[Validovať analýzu] simulácia, replikácia (bootstrap)
		\end{description}
		Nie presne v tomto poradí, budeme sa hýbať v kruhoch.
	\end{block}
\end{frame}

\begin{frame}{Čo sa naučíte v tomto kurze}
	\begin{block}{2. Používať moderný ekosystém pre prácu s dátami}
		\begin{description}
			\item[Open source] software, nepotrebujeme nakupovať drahý štatistický software, ani MS Office, všetko čo potrebujeme sa dá stiahnuť z Internetu.
			\item[Version control] Chceme, aby sa dáta dali zdieľať a boli chránené pred stratou alebo nechcenou zmenou.
			\item[Rôzne zdroje dát] Chceme pracovať s dátami z viacerých možných zdrojov - textové súbory, Excel, JSON, databázy atď.
			\item[Zdieľanie dát a analýzy] Chceme, aby si ľudia mohli skontrolovať našu analýzu.
		\end{description}
	\end{block}
\end{frame}


% ================================================================
% Outline
% ================================================================

\begin{frame}{Outline}
\tableofcontents
\end{frame}

% ================================================================
\section{Čo potrebujete pre tento kurz}
% ================================================================

% ----------------------------------------------------------------
\subsection{Znalosti}
% ----------------------------------------------------------------
\begin{frame}{Čo sa očakáva, že budete vedieť}
	\begin{block}{Základné znalosti z pravdepodobnosti a štatistiky}
		\begin{itemize}
			\item Stačí, aby ste sa veľmi nezľakli, keď poviem t-test.
			\item Máte určitú prax v spracovaní a zobrazovaní dát.
		\end{itemize}
	\end{block}
	\begin{block}{Programovanie}
		\begin{itemize}
			\item Očakávam, že máte za sebou kurz programovania, v hocičom.
			\item Napríklad keď idete písať kód, začnete tým, že si zapnete anglickú klávesnicu.
			\item A pochytili ste trocha algoritmického myslenia. 
			\item Budeme sa učiť nový jazyk a používať nové nástroje, takže pôjdeme od nuly.
		\end{itemize}
	\end{block}
	\begin{block}{Angličtina}
		\begin{itemize}
			\item Je mi ľúto, ale bez angličtiny budete mať v tomto kurze ťažkosti.
			\item Predovšetkým si veľmi ťažko budete hľadať pomoc na Internete, a to je prvá vec, ktorú človek robí, keď mu niečo nefunguje alebo nevie, ako niečo urobiť. 
		\end{itemize}
	\end{block}
\end{frame}

% ----------------------------------------------------------------
\subsection{Laptop}
% ----------------------------------------------------------------
\begin{frame}{Ešte potrebujete laptop}
	\begin{block}{Laptop}
		\begin{itemize}
			\item Windows 7 alebo 10, alebo Linux
			\begin{itemize}
				\item Windows 10 má WSL - Windows Subsystem for Linux - a umožňuje vám lepšie používať niektoré veci, napríklad \texttt{git}.
				\item Ale Linux je na to ešte lepší.
			\end{itemize}
			\item Nepotrebujete mať extra silný procesor alebo veľa pamäti, aspoň nie pre tento kurz.
		\end{itemize}
	\end{block}
	\begin{block}{Inštalovaný software}
		\begin{itemize}
			\item Potrebujete mať nainštalované R a RStudio.
			\item Zriaďte si účet na GitHub (\texttt{https://www.github.org}) a stiahnite si \emph{GitHub Desktop}. Ak máte Linux, stačí vám nainštalovať git. 
			\begin{itemize}
				\item Toto stačí v priebehu kurzu, svoj GitHub account budete potrebovať na odovzdanie zadaní.
				\item Ale nezaškodí získať trocha praxe, takže čím skôr, tým lepšie.
			\end{itemize}
			\item Návod na inštaláciu R/RStudio a na zriadenie GitHub konta nájdete v GitHub repozitári tohoto kurzu, \texttt{https://www.github.com/PKvasnick/RTutorial/}
			\item Ak by ste mali problémy s inštaláciou, rýchlo sa ozvite.
		\end{itemize}
	\end{block}
\end{frame}

% ----------------------------------------------------------------
\subsection{Informácie}
% ----------------------------------------------------------------
\begin{frame}{Kde hľadať informácie}
	\begin{block}{Príručky}
		Príručiek je veľa, väčšina aktuálnych a moderných je v angličtime.
		\begin{itemize}
			\item Na Pinterestovej stránke \texttt{https://sk.pinterest.com/peterkvasnika/my\_r/} nájdete odkazy na niekoľko internetových portálov a PDF dokumentov, ktoré vám môžu pomôcť v začiatkoch.
			\begin{itemize}
				\item Niekoľko z nich je slovenčine/češtine.
			\end{itemize}
			\item Na portáli CRAN (Comprehensive R-Archive Network - \texttt{https://cran.r-project.org/}) nájdete prehľad dokumentácie k R.
		\end{itemize}
	\end{block}
	\begin{block}{Nápoveď v R a RStudio}
		\begin{itemize}
			\item R má svoj vlastný help systém, naučíte sa s ním pracovať.
			\item RStudio má takisto svoje helpy.
		\end{itemize}
	\end{block}
\end{frame}

\begin{frame}{Kde hľadať informácie}
	\begin{block}{Internet}
		\begin{itemize}
			\item To čo programátor robí najčastejšie je, že vysvetlí Googlu lámanou angličtinou čo chce urobiť (\texttt{R create dataframe}), alebo priamo do riadku vyhľadávača skopíruje chybovú hlášku. 
			\item S vysokou pravdepodobnosťou nájdete použiteľnú odpoveď, či už je vaša otázka triviálna alebo zložitá.
			\begin{itemize}
				\item Tú odpoveď nájdete najčastejšie na webe StackOverflow, \texttt{https://stackoverflow.com}, s ktorým sa určite spriatelíte.
			\end{itemize}
			\item Časom prídete na to, že kúsok fungujúceho kódu býva užitočnejší ako podrobný výklad syntaxe. 
		\end{itemize}
	\end{block}
\end{frame}
 \clearpage
\clearpage % ================================================================
\section{R}
% ================================================================
\begin{frame}{Prečo R?}
	\begin{block}{Máme predsa ...}
		\begin{description}
			\item[Excel] a iné tabuľkové programy
			\item[SPSS] Statisticu, Minitab a iné komerčné programy poskytujúce analýzu na kľúč
		\end{description}
		\alert{Tak prečo mám používať niečo, čo sa treba určitý čas učiť?}
	\end{block}
\end{frame}

\begin{frame}{Niekoľko dôvodov}
	\begin{block}{Excel nie je štatistický program}
		\begin{itemize}
			\item Excel je výborný nástroj na vkladanie dát, získavanie a konsolidáciu dát z databáz a na základné úpravy dát
			\item Ale nie je dobrý na výmenu dát (polo-proprietárny formát - nikdy nevieme, kedy sa zmení)
			\item Vzorce v bunkách sa ťažko spravujú a neexistuje praktický spôsob, ako nezávisle dokumentovať, čo sa ako počíta.
			\item Nemáme výstrahu, ak náhodne zmeníme obsah bunky
			\item Nástroje pre štatistiku sú implementované ledabylo.
			\item Grafy sú na zaplakanie.
		\end{itemize}
	\end{block}
\end{frame}

\begin{frame}{Niekoľko dôvodov}
	\begin{block}{Robíme stále zložitejšie analýzy}
		\begin{itemize}
			\item Chceme skúmať analyzovať zložité a veľké dáta
			\item Chceme validovať našu analýzu pomocou simulácií a replikácie - potrebujeme analýzy opakovať tisíckrát
			\item Chceme formulovať a testovať zložité modely (\emph{Data Science})
		\end{itemize}
	\end{block}
	\begin{block}{Chceme zdieľať dáta a analýzu}
		\begin{itemize}
			\item Potrebujeme otvorený software a nie drahé štatistické balíky alebo MS Office
			\item Potrebujeme otvorené formáty dát
			\item Chceme software, ktorý sa \emph{rýchlo inovuje}
			\item Chceme software, ktorý je správny
		\end{itemize}
	\end{block}
\end{frame}

\begin{frame}{Preto chceme R!}
	\begin{block}{R je programovate2né}
		\begin{itemize}
			\item R je interpretovaný programovací jazyk
			\item R podporuje integráciu s inými programovacími jazykmi - môžeme volať funkcie naprogramované v C++, Fortrane ap., čo podstatne kompenzuje pomalosť vlastného interpreta R.
			\item R spolupracuje s Pythonom, Javou a ďalšími jazykmi, ktoré používajú vývojári v data science
			\item R má výborné IDE, RStudio, a najnovšie aj Visual Studio!
		\end{itemize}
	\end{block}
	\begin{block}{R má bohaté rozhrania pre dáta}
		\begin{itemize}
			\item R umožňuje čítať dáta z veľkého množstva vstupných formátov:
			\begin{itemize}
				\item textových súborov
				\item Excelu
				\item JSON
				\item databáz
				\item Apache Spark-u
				\item \dots
			\end{itemize}
			\item R dokáže dáta, grafy a reporty exportovať do veľkého množstva formátov.
		\end{itemize}
	\end{block}
\end{frame}

\begin{frame}{Preto chceme R!}
	\begin{block}{R je rozšíriteľné}
		\begin{itemize}
			\item To, čo robí R skutočne cenným, je ekosystém rozšírení - baličkov (packages)
			\item Tieto baličky obsahujú všetky štatistické metódy, ktoré kedy budete potrebovať
			\item Balíčky sídlia na serveri CRAN, a môžete si ich ľahko doinštalovať cez interpret R (\texttt{install.packages(<menobalíčka>}
			\item Balíčky neustále pribúdajú: Ak niekto opublikuje novú štatistickú metódu, s veľkou pravdepodobnosťou ju hneď implementuje v R.
		\end{itemize}
	\end{block}
	\begin{block}{R je renomované a spoľahlivé}
		\begin{itemize}
			\item Pretože R používa veľa ľudí, je dobre otestované a všetky prípadné chyby sú hneď odstránené.
			\item Ak si svoje dáta analyzujete v R, nikto sa nebude pýtať, či ste správne počítali ANOVu.
			\item Kód vašej analýzy je univerzálne zrozumiteľný doklad o tom, čo ste robili.
		\end{itemize}
	\end{block}
\end{frame}

\begin{frame}{Nemá chybu...?}
	\begin{block}{R má svoje špecifiká a slabé stránky}
		\begin{itemize}
			\item R sa pôvodne vyvinulo z funkcionálneho a objektovo-orientovaného jazyka S. Preto niektoré veci pracujú trocha odlišne.
			\item Pretože R má za sebou dlhú históriu, obsahuje niekoľko súperiacich koncepcií a funkčných rozhraní. Preto niektoré veci možno robiť rôznymi spôsobmi, a naopak niektoré podobné veci musíte robiť odlišne.
			\item R je pomalé. Treba sa vyhýbať zložitým programovým konštrukciám v R (cyklom \texttt{for} a podobne), a používať čo najviac metafunkcie R (\texttt{apply}), aby sa počítanie robilo v C a Fortrane, a nie v R. 
			\item Napriek tomuto všetkému sa základy programovania v R možno naučiť pomerne rýchlo a pomerne rýchlo získať výsledky.
		\end{itemize}
	\end{block}
\end{frame}
 \clearpage

\begin{frame}{Ideme na to...}
	\begin{itemize}
		\item Otvorte si v prehliadači stránku \texttt{https://www.github.com/PKvasnick/RTutorial/} a stiahnite si z adresára \texttt{code} všetky súbory \texttt{*.Rmd}.
		\begin{itemize}
			\item Preklikajte sa k súboru a zvoľte \emph{Raw} zobrazenie.
			\item Pravý klik a \emph{Save As...}. Je to čisto textový súbor.
		\end{itemize}
		\item Odporúčam vytvoriť si podobnú adresárovú štruktúru ako v mojom repozitári.
		\item (Úplne najlepšie) Môžete si tiež vytvoriť klon môjho repozitára pomocou GitHub Desktopu.
		\item Spustite RStudio, \texttt{File->Open...} a nájdite súbor .\texttt{R01\_PrveKroky.Rmd}
	\end{itemize}
\end{frame}

\end{document}